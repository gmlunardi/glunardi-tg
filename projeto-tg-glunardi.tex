%%=============================================================================
%%
%% Exemplo de Projeto de TG seguindo a formatação MDT UFSM
%% Vitor da Silva
%% vitords@inf.ufsm.br
%%
%% A classe mdtufsm.cls não foi criada por mim, apenas modificada para 
%% facilitar a criação de projetos de TG (projtg).
%% O original está disponível em: http://code.google.com/p/mdtufsm-ppgi/
%% A classe esta de acordo com o MDT 2010, porém não houveram alterações
%% na formatação entre as versões 2010 e 2012 (atual), apenas correções no 
%% texto e algumas definições para impressão frente e númeração de páginas
%% para trabalhos com mais de 100 páginas.
%% Dito isso, NÃO GARANTO que a classe esteja totalmente de acordo com
%% qualquer versão do MDT.
%%
%% Opções para a classe do documento:
%%  - projtg: Projeto de TG
%%  - tg: Trabalho de Graduação
%%  - espec: Monografia de Especialização
%%  - tese: Tese de Doutorado
%%  - diss: Dissertação de Mestrado
%%
%% Este documento e um exemplo de Projeto de TG, para outras funcionalidades
%% olhar no mdtufsm.cls :)
%%
%% É necessário seguir esta sequência para compilar o documento:
%%	- compilar LaTeX
%%	- compilar BibTeX
%%	- compilar LaTeX
%%	- compilar LaTeX
%%
%% O warning "Underfull \vbox (badness 10000) detected at line XX" pode ser
%% ignorado sem problemas.
%%=============================================================================

\documentclass[projtg]{mdtufsm}

\usepackage[T1]{fontenc}
\usepackage[utf8]{inputenc}
% para funcionar corretamente o tamanho das fontes da capa
\usepackage{fix-cm}
% pacote para usar fonte Adobe Times e cores			
\usepackage{times, color, xcolor}
% pacote para importar figuras
\usepackage{graphicx}
% pacotes matemáticos
\usepackage{amsmath, latexsym, amssymb}
\usepackage{tabu}
\usepackage[inner=30mm, outer=20mm, top=30mm, bottom=20mm]{geometry} 
% só é necessário para o lorem ipsum
\usepackage{lipsum}
% hidelinks disponível no pacote hyperref a partir da versão 2011-02-05  6.82a
\usepackage[%hidelinks%, 
            bookmarksopen=true,linktoc=none,colorlinks=true,
            linkcolor=black,citecolor=black,filecolor=magenta,urlcolor=blue,
            pdftitle={Título do Trabalho},
            pdfauthor={Autor},
            pdfsubject={Assunto (TG, Projeto, etc.)},
            pdfkeywords={keyword1, keyword2}
            ]{hyperref}

\input{macros/bugcaption}

\title{Mineração de dados em dados do curso Pró-Conselho/UFSM para a identificação do perfil dos conselhos municipais de educação do RS}

\author{Machado Lunardi}{Gabriel}
\course{Sistemas de Informação}
\altcourse{curso de Sistemas de Informação}
\institute{Centro de Tecnologia}
\degree{Bacharel em Sistemas de Informação}
\trabalhoNumero{}
\advisor[Prof.ª]{Dr.ª}{Winck}{Ana T.}

\date{15}{Agosto}{2014}

% Exemplos de keywords
\keyword{Classificação Multirrótulo} 
\keyword{Aprendizado de Máquina}
\keyword{Mineração de Dados}

\begin{document}
\maketitle
\setlength{\baselineskip}{1.5\baselineskip}

\chapter{Introdução}

O Programa do Pró-Conselho é proposto pelo Ministério da Educação (MEC), consiste em um Curso de Extensão a Distância de Formação Continuada de Conselheiros Municipais de Educação. O curso configura-se como iniciativa da Secretaria de Educação Básica (SEB) que visa fortalecer os Sistemas de Ensino e as instâncias políticas e sociais tal como é o Conselho Municipal de Educação (CME). Tem carga horária de 160h, ofertado via internet, em ambiente virtual de aprendizagem (plataforma Moodle) ministrado pela Universidade Federal de Santa Maria em parceria com a Coordenação do Programa Nacional de Capacitação de Conselheiros Municipais de Educação – Pró-Conselho, SEB/MEC. Dentre os conteúdos discutidos e estudados estão: Educação e Tecnologia, Concepção, Estrutura e Funcionamento do CME, Conselho Municipal e as Políticas Públicas. O público alvo é formado por Conselheiros Municipais de Educação e técnicos das secretarias de educação dos municípios onde ainda não existam Conselhos Municipais de Educação. 

 O Pró-Conselho/UFSM é responsável pela oferta do curso no estado do Rio Grande do Sul - RS, lotado no Centro de Educação da UFSM, estando em sua 2ª edição. A equipe é composta por: coordenação geral; coordenação adjunta local responsável por contatar cursistas (alunos) pessoalmente ou por telefone; coordenação adjunta pedagógica para resolver questões de aprendizado; um professor supervisor com o intuito de orientar e guiar a tutoria; tutores que agem diretamente com os cursistas; apoio de Informática encarregado da gestão tecnológica; apoio administrativo incumbido das tratativas administrativas. Vale ressaltar que o projeto está diretamente ligado aos grupos de pesquisa dessa área, pois serve como objeto de estudo. 

Nessa perspectiva, como uma das atividades do curso, a equipe gestora formulou um questionário composto por perguntas abertas e fechadas buscando delinear o perfil dos CMEs no RS. Tal questionário foi utilizado como requisito inicial no desenvolvimento de um banco de dados o qual receberá as respostas. A partir daí, um sistema {\it web}, a ser desenvolvido, funcionará como instrumento de coleta e, após análise, visualização dos dados. É importante destacar que existe um sistema, desde 2003, para o mesmo fim, de âmbito nacional, chamado Sistema de Informações dos Conselhos Municipais de Educação (SICME). No entanto, o sistema encontra-se em manutenção segundo consta no portal eletrônico do MEC.

Dentre as tecnologias para análise de dados, destaca-se a mineração de dados. A mineração de dados {\it Data Mining} é uma etapa do processo de KDD {\it Knowledge Discovery in Databases} - Descoberta de Conhecimento em Bases de Dados buscando por padrões em grandes bases de dados. Para \cite{Han-kamber2nd} mineração de dados é o processo de proposição de várias consultas, extração de informações úteis, padrões e tendências a partir de grande quantidade de dados armazenada em banco de dados. 

Tendo em vista o conhecimento sobre a mineração de dados e desenvolvimento de sistemas, objetiva-se construir um sistema {\it web} capaz de servir para a coleta, gerenciamento, visualização e a análise os dados buscando delinear o perfil dos CMEs no RS.

A escolha do contexto do trabalho justfica-se através dos fatores: inatividade do SICME e suas incongruências devido a sua grande abrangência; necessidade de um recurso computacional, por parte da equipe, para coletar e analisar os dados de forma automatizada e a oportunidade de consolidar conhecimentos em mineração de dados em um estudo de caso real.

Como consequência deste trabalho pretende-se contribuir com uma metodologia de análise e visualização de dados automatizada para o Pró-Conselho/UFSM. Com isso, busca-se fornecer informação para o embasamento de novas leis que reforcem os CME, revertendo assim, o bem da educação para a sociedade. 

\chapter{Objetivos}
\section{Objetivo Geral}

Empregar técnicas de mineração de dados em um banco de dados do curso Pró-Conselho/UFSM no intuito de identificar o perfil dos conselhos municipais de educação no Rio Grande do Sul. 
	
\section{Objetivos Específicos}
\begin{itemize}
  \item Desenvolver um sistema {\it web} para a coleta e visualização dos dados.
  \item Pesquisar sobre as técnicas de mineração de dados;
  \item Estudar formas de pré-processar o dataset obtido;
  \item Identificar e aplicar uma técnica de mineração de dados que melhor se enquadre sobre os dados resultantes do pré-processamento;
  \item Encontrar o perfil dos conselhos municipais de educação no RS.
\end{itemize}

\chapter{Justificativa}
iniciativa para outros estados brasileiros de forma a descentralizar o SICME

\chapter{Revisão de Literatura}


\chapter{Metodologia}
O início do trabalho se dará através do estudo e revisão de literatura de alguns assuntos: técnicas de mineração e pré-processamento de dados. Concomitante a isso, desenvolver um sistema {\it web} responsável por receber os dados fornecidos pelos cursistas. Na sequência, com o banco de dados populado, conhecer o dataset obtido na intenção de realizar o pré-processamento da melhor forma possível. Por fim, aplicar a técnica de mineração de dados escolhida e averiguar o conhecimento obtido. 
    
\chapter{Cronograma}
\begin{tabu} to 0.9\linewidth{|X[6]|X|X|X|X|X|}
	\hline
	& AGO & SET & OUT & NOV & DEZ \\
	\hline
	Definição do projeto & X & & & & \\
	\hline
	Desenvolvimento do sistema web de coleta & X & & & & \\
	\hline
	Pesquisa sobre as técnicas de mineração de dados & X & X & & & \\
	\hline
	Coleta das respostas através do questionário on-line & & X & & & \\
	\hline
	Identificação e aplicação da técnica de mineração escolhida & & X & X & & \\
	\hline
	Revisão de Literatura & X & X & X & & \\
	\hline
	Apresentação do andamento do TG & & & X & & \\
	\hline
	Elaboração da parte escrita & & & X & X & \\
	\hline
	Reuniões de acompanhamento & X & X & X & X & X \\
	\hline
	Apresentação do TG para a banca & & & & & X \\
	\hline
\end{tabu}
 
\setlength{\baselineskip}{\baselineskip}

% Referências
\bibliographystyle{abnt}
\bibliography{referencias}

\end{document}
