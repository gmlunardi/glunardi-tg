%%=============================================================================
%%
%% Exemplo de Projeto de TG seguindo a formatação MDT UFSM
%% Vitor da Silva
%% vitords@inf.ufsm.br
%%
%% A classe mdtufsm.cls não foi criada por mim, apenas modificada para 
%% facilitar a criação de projetos de TG (projtg).
%% O original está disponível em: http://code.google.com/p/mdtufsm-ppgi/
%% A classe esta de acordo com o MDT 2010, porém não houveram alterações
%% na formatação entre as versões 2010 e 2012 (atual), apenas correções no 
%% texto e algumas definições para impressão frente e númeração de páginas
%% para trabalhos com mais de 100 páginas.
%% Dito isso, NÃO GARANTO que a classe esteja totalmente de acordo com
%% qualquer versão do MDT.
%%
%% Opções para a classe do documento:
%%  - projtg: Projeto de TG
%%  - tg: Trabalho de Graduação
%%  - espec: Monografia de Especialização
%%  - tese: Tese de Doutorado
%%  - diss: Dissertação de Mestrado
%%
%% Este documento e um exemplo de Projeto de TG, para outras funcionalidades
%% olhar no mdtufsm.cls :)
%%
%% É necessário seguir esta sequência para compilar o documento:
%%	- compilar LaTeX
%%	- compilar BibTeX
%%	- compilar LaTeX
%%	- compilar LaTeX
%%
%% O warning "Underfull \vbox (badness 10000) detected at line XX" pode ser
%% ignorado sem problemas.
%%=============================================================================

\documentclass[projtg]{mdtufsm}

\usepackage[T1]{fontenc}
\usepackage[utf8]{inputenc}
% para funcionar corretamente o tamanho das fontes da capa
\usepackage{fix-cm}
% pacote para usar fonte Adobe Times e cores			
\usepackage{times, color, xcolor}
% pacote para importar figuras
\usepackage{graphicx}
% pacotes matemáticos
\usepackage{amsmath, latexsym, amssymb}
\usepackage{tabu}
\usepackage[inner=30mm, outer=20mm, top=30mm, bottom=20mm]{geometry} 
% só é necessário para o lorem ipsum
\usepackage{lipsum}
% hidelinks disponível no pacote hyperref a partir da versão 2011-02-05  6.82a
\usepackage[%hidelinks%, 
            bookmarksopen=true,linktoc=none,colorlinks=true,
            linkcolor=black,citecolor=black,filecolor=magenta,urlcolor=blue,
            pdftitle={Título do Trabalho},
            pdfauthor={Autor},
            pdfsubject={Assunto (TG, Projeto, etc.)},
            pdfkeywords={keyword1, keyword2}
            ]{hyperref}


%%=============================================================================
%% Trampa para corrigir o bug do hyperref que redefine o caption das figuras e das
%% tabelas, n�o colocando o nome ``Figura'' antes do n�mero do mesmo na lista
%%=============================================================================

\makeatletter

\long\def\@caption#1[#2]#3{%
  \expandafter\ifx\csname if@capstart\expandafter\endcsname
                  \csname iftrue\endcsname
    \global\let\@currentHref\hc@currentHref
  \else
    \hyper@makecurrent{\@captype}%
  \fi
  \@ifundefined{NR@gettitle}{%
    \def\@currentlabelname{#2}%
  }{%
    \NR@gettitle{#2}%
  }%
  \par\addcontentsline{\csname ext@#1\endcsname}{#1}{%
    \protect\numberline{\csname fnum@#1\endcsname ~-- }{\ignorespaces #2}%
  }%
  \begingroup
    \@parboxrestore
    \if@minipage
      \@setminipage
    \fi
    \normalsize
    \expandafter\ifx\csname if@capstart\expandafter\endcsname
                    \csname iftrue\endcsname
      \global\@capstartfalse
      \@makecaption{\csname fnum@#1\endcsname}{\ignorespaces#3}%
    \else
      \@makecaption{\csname fnum@#1\endcsname}{%
        \ignorespaces
        \ifHy@nesting
          \expandafter\hyper@@anchor\expandafter{\@currentHref}{#3}%
        \else
          \Hy@raisedlink{%
            \expandafter\hyper@@anchor\expandafter{%
              \@currentHref
            }{\relax}%
          }%
          #3%
        \fi
      }%
    \fi
    \par
  \endgroup
}

\makeatother

\title{Mineração de dados em dados do curso Pró-Conselho/UFSM para a identificação do perfil dos conselhos municipais de educação do RS}

\author{Machado Lunardi}{Gabriel}
\course{Sistemas de Informação}
\altcourse{curso de Sistemas de Informação}
\institute{Centro de Tecnologia}
\degree{Bacharel em Sistemas de Informação}
\trabalhoNumero{}
\advisor[Prof.ª]{Dr.ª}{Winck}{Ana T.}

\date{15}{Agosto}{2014}

% Exemplos de keywords
\keyword{Aprendizado de Máquina}
\keyword{Mineração de Dados}

\begin{document}
\maketitle
\setlength{\baselineskip}{1.5\baselineskip}

\chapter{Introdução}

O Programa do Pró-Conselho é proposto pelo Ministério da Educação (MEC), consiste em um Curso de Extensão a Distância de Formação Continuada de Conselheiros Municipais de Educação. Segundo \cite{Lunardi-2014}, o curso configura-se como iniciativa da Secretaria de Educação Básica (SEB) que visa fortalecer os Sistemas de Ensino e as instâncias políticas e sociais tal como é o Conselho Municipal de Educação (CME). Tem carga horária de 160h, ofertado via internet, em ambiente virtual de aprendizagem (plataforma \textit{Moodle}) ministrado pela Universidade Federal de Santa Maria em parceria com a Coordenação do Programa Nacional de Capacitação de Conselheiros Municipais de Educação – Pró-Conselho, SEB/MEC. O público alvo é formado por Conselheiros Municipais de Educação e técnicos das secretarias de educação dos municípios onde ainda não existam Conselhos Municipais de Educação. 

O Pró-Conselho/UFSM é responsável pela oferta do curso no estado do Rio Grande do Sul - RS, lotado no Centro de Educação da UFSM, estando em sua 2ª edição. A equipe é composta por: coordenação geral; coordenação adjunta local; coordenação adjunta pedagógica; um professor supervisor; tutores; apoio de Informática e apoio administrativo. Vale ressaltar que o projeto está diretamente ligado aos grupos de pesquisa dessa área, pois serve como objeto de estudo. 

Nessa perspectiva, como uma das atividades do curso, a equipe gestora formulou um questionário composto por perguntas abertas e fechadas buscando delinear o perfil dos CME no RS. Tal questionário foi utilizado como requisito inicial no desenvolvimento de um banco de dados relacional o qual receberá as respostas. A partir daí, um sistema {\it web}, a ser desenvolvido, funcionará como instrumento de coleta e, após análise, visualização dos dados. É importante destacar que existe um sistema, desde 2003, com proposição parecida, porém de âmbito nacional, chamado Sistema de Informações dos Conselhos Municipais de Educação \cite{sicme-site}. Todavia, conforme consta no site \cite{mec-site} o sistema encontra-se em manutenção. 

Sendo assim, como tema central, empregar-se-á técnicas de mineração de dados buscando padrões que auxiliem a identificar o perfil dos CME no RS através dos dados coletados via sistema {\it on-line}. Ao identificar o perfil, entende-se ser possível explorar a realidade dos CME e, por conseguinte, aplicar políticas e ações que melhorem tais órgãos fortalecendo o sistema de educação básica.

% escolha desse contexto justifica-se, principalmente, pela necessidade em conhecer a realidade dos CME de um estado e, por conseguinte, aplicar melhores políticas   do SICME e suas incongruências devido a sua grande abrangência; necessidade de um recurso computacional, por parte da equipe, para coletar e analisar os dados de forma automatizada e a oportunidade de consolidar conhecimentos em mineração de dados em um estudo de caso real.

%Como consequência deste trabalho pretende-se contribuir com uma metodologia de análise e visualização de dados automatizada para o Pró-Conselho/UFSM. Com isso, busca-se fornecer conhecimento para o embasamento de novas leis que reforcem os CME, revertendo assim, o bem da educação para a sociedade. 

\chapter{Objetivos}
\section{Objetivo Geral}

Empregar técnicas de mineração de dados, em um banco de dados relacional do curso Pró-Conselho/UFSM, buscando padrões que auxiliem na identificação do perfil e realidade dos CME no RS.
	
\section{Objetivos Específicos}
\begin{itemize}
  \item Construir um banco de dados relacional e populá-lo;
  \item Desenvolver um sistema {\it web} para a coleta e visualização dos dados.
  \item Pesquisar sobre as técnicas de mineração de dados;
  \item Estudar formas de pré-processar o \textit{data set} obtido;
  \item Identificar e aplicar uma técnica de mineração de dados que melhor se enquadre sobre os dados resultantes do pré-processamento;
  \item Encontrar padrões que auxiliem na identificação do perfil dos conselhos municipais de educação no RS.
\end{itemize}

\chapter{Justificativa}
As atividades cotidianas de qualquer pessoa implicam na geração de dados, os quais são armazenados por muitas instituições sumarizando grandes quantidades. Entretanto, muitas dessas instituições não atingem a transformação dos dados em conhecimento permitindo melhorar a qualidade de vida das pessoas. 

Nesse sentido, com a identificação do perfil dos conselhos municipais de educação no RS, através da mineração de dados, será possível gerar conhecimento sobre a realidade dos CME. A partir disso, alcançar-se-á o refinamento do Pró-Conselho/UFSM e, consequentemente da instrumentalização dos cursistas que, por sua vez, estarão melhor preparados para implementar os conselhos e, principalmente, contribuir para a qualidade do ensino básico brasileiro.   

%iniciativa para outros estados brasileiros de forma a descentralizar o SICME ....

\chapter{Revisão de Literatura}

\section{Mineração de Dados}

Na atualidade são gerados muitos dados, principalmente em em razão da rápida evolução computacional, onde destaca-se a disseminação da internet. Conforme mostra \cite{enia5}, esse grande volume de dados é armazenado em repositórios de diferentes áreas, seja médica, científica, financeira, comercial, dentre muitas outras que nos circundam. A partir disso, informações úteis podem ser extraídas dos dados, porém torna-se inviável extraí-las e interpretá-las manualmente, utilizando só a subjetividade humana. É nesse sentido, em automatizar parte do processo, que existem técnicas computacionais adequadas, em especial as relacionadas a mineração de dados.

Tendo em mente o exposto, a mineração de dados, em consonância com \cite{hand}, é "a análise de grandes conjuntos de dados a fim de encontrar relacionamentos inesperados e de resumir os dados de uma forma que eles sejam tanto úteis enquanto compreensíveis ao dono dos dados".

Já na perspectiva de \cite{cabena}, a mineração de dados é "um campo multidisciplinar que reúne técnicas de aprendizagem de máquina, reconhecimento de padrões, estatísticas, banco de dados e visualização, para conseguir extrair informações de grandes bases de dados.". 

Neste trabalho entende-se mineração de dados como uma etapa de um processo maior denominado KDD ({\it Knowledge Discovery in Databases} - Descoberta de Conhecimento em Bases de Dados), pois existem etapas, anteriores e posteriores a mineração, muito importantes. Alguns defensores desse ponto de vista são \cite{fayyad} que reafirmam "[...] KDD refere-se a todo o processo de descoberta de conhecimento e que mineração de dados compreende a aplicação de algoritmos específicos para a extração de padrões". No entanto, na literatura, ainda não é consenso da definição de mineração de dados e KDD, porém há a consonância de que o procedimento de mineração deva ser iterativo, interativo e dividido em fases.  

Vale mencionar que mineração de dados não compreende uma simples consulta em um banco de dados ou, então, uma pesquisa na \textit{web}. Essas ações pertencem a área de recuperação de informação (\textit{information retrieval}) e não devem ser confundidas com mineração de dados, como verificado por \cite{pang-ning}.

\section{Tarefas da Mineração de Dados}

As tarefas de mineração de dados são dividias em duas categorias: preditivas e descritivas. As preditivas preveem o valor de um atributo com base nos valores de outros atributos, são elas: classificação e estimação. Já as tarefas descritivas visam a derivação de padrões (correlações, tendências, grupos) que sintetizem os relacionamentos subjacentes nos dados, sendo: regras de associação, agrupamento, dentre outras \cite{pang-ning}. Assim, \cite{larose2005} elenca as principais tarefas conforme explicado.

\textbf{Classificação (Classification):} identifica a qual classe (categoria) pertence um determinado atributo. O conjunto de registros é analisado, sendo que cada um deles já possui a identificação da classe, na intenção de "aprender" como classificar um novo registro (aprendizado supervisionado).

\textbf{Estimação (Estimation):} é semelhante à classificação, no entanto é usada quando o atributo alvo é numérico ao invés de categórico. Basicamente novas observações são baseadas em um modelo de estimação que, por sua vez, advém dos valores de outras variáveis. Por exemplo, estimar a quantia a ser gasta por uma família no período de férias. 

\textbf{Agrupamento (Clustering):} objetiva aproximar registros, observações ou classes de objetos similares. Diferentemente das tarefas anteriores, o agrupamento não focaliza o valor de uma variável alvo. Em vez disso, algoritmos de agrupamento procuram segmentos definidos em subgrupos ou grupos relativamente homogêneos, onde a similaridade dos registros dentro do cluster é maximizada e a similaridade fora é minimizada. Alguns exemplos, nas mais variadas áreas, são: processamento de imagens, pesquisa da mercado, reconhecimento de padrões. 

\textbf{Associação (Association):} visa descobrir as regras para quantificar a relação entre dois ou mais atributos. As regras de associação são da forma "SE atributo X ENTÃO atributo Y". Por exemplo, de 1000 clientes em um dia de semana a noite, de uma loja, 200 compraram fraldas e, desses 200, 50 compraram cervejas. Logo, a regra de associação poderia ser "Se comprou fraldas, então comprará cervejas".


\section{Pré-processamento dos dados}

O conhecimento obtido após o término oficial de um processo de KDD está intimamente ligado a qualidade dos dados de entrada. Todavia, se os dados presentes em grandes bases de dados e \textit{data warehouses} reais são inconsistentes, incompletos e com ruídos, conforme elenca \cite{Han-kamber2nd}, então o conhecimento será de má qualidade. Nesse sentido, \cite{larose2005} diz que "para ser útil para fins da mineração de dados, os dados precisam passar por pré-processamento, na forma de limpeza e transformação de dados.".

O processo de preparação dos dados para a mineração, denominado de pré-processamento por \cite{Han-kamber2nd}, compreende as principais etapas:

\textbf{Limpeza dos dados:} Dados tendem a ser incompletos e inconsistentes, assim, essa etapa visa a aplicação de métodos como, por exemplo, preenchimento de campos vazios com valores padrão e agrupamento buscando descobrir melhores valores, para a eliminação de problemas. 

\textbf{Integração dos dados:} Como o próprio nome sugere, integração compreende a etapa de integrar diferentes fontes de armazenamento de dados em um único repositório consistente. A partir disso, deve-se observar questões de redundância e dependência existentes entre atributos.

\textbf{Transformação dos dados:} Essa etapa merece atenção, pois alguns algoritmos trabalham apenas com valores numéricos e outros com valores categóricos. Sendo assim, é necessária a transformação entre valores categóricos/numéricos de acordo com os objetivos pretendidos. Algumas etapas podem ser citadas para esse fim: suavização de dados discrepantes \textit{(outliers)}, agrupamento, generalização, normalização e a criação de novos atributos.

\textbf{Redução dos dados:} Um \textit{data set} pode ser muito grande ocasionando demora e até inviabilidade no processamento e análise de dados. Assim, essa etapa busca diminuir o volume de um \textit{data set} através da remoção de atributos que não denigram a integridade dos dados originais. Com isso, minerar em dados reduzidos seria mais rápido e, ainda, produziria o mesmo resultado. 




\chapter{Metodologia}
O início do trabalho se dará através do estudo e revisão de literatura de alguns assuntos: técnicas de mineração e pré-processamento de dados. Concomitante a isso, o desenvolvimento de um sistema {\it web} responsável por receber os dados fornecidos pelos cursistas. Tais dados serão armazenados em um banco de dados relacional, a ser construído. Na sequência, com o banco de dados populado, conhecer o \textit{data set} obtido na intenção de realizar o pré-processamento da melhor forma possível. Por fim, aplicar a técnica de mineração de dados escolhida e averiguar o conhecimento obtido. 
    
\chapter{Cronograma}
\begin{tabu} to 0.9\linewidth{|X[6]|X|X|X|X|X|}
	\hline
	& AGO & SET & OUT & NOV & DEZ \\
	\hline
	Definição do projeto & X & & & & \\
	\hline
	Construção do banco de dados relacional & X & & & & \\
	\hline
	Desenvolvimento do sistema web de coleta & X & & & & \\
	\hline
	Pesquisa sobre as técnicas de mineração de dados & X & X & & & \\
	\hline
	Coleta das respostas através do questionário on-line & & X & & & \\
	\hline
	Identificação e aplicação da técnica de mineração escolhida & & X & X & & \\
	\hline
	Revisão de Literatura & X & X & X & & \\
	\hline
	Apresentação do andamento & & & X & & \\
	\hline
	Elaboração da parte escrita & & X & X & X & \\
	\hline
	Reuniões de acompanhamento & X & X & X & X & X \\
	\hline
	Apresentação para a banca & & & & & X \\
	\hline
\end{tabu}
 
\setlength{\baselineskip}{\baselineskip}

% Referências
\bibliographystyle{abnt}
\bibliography{referencias}

\end{document}
