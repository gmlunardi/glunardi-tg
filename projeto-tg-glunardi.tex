%%=============================================================================
%%
%% Exemplo de Projeto de TG seguindo a formatação MDT UFSM
%% Vitor da Silva
%% vitords@inf.ufsm.br
%%
%% A classe mdtufsm.cls não foi criada por mim, apenas modificada para 
%% facilitar a criação de projetos de TG (projtg).
%% O original está disponível em: http://code.google.com/p/mdtufsm-ppgi/
%% A classe esta de acordo com o MDT 2010, porém não houveram alterações
%% na formatação entre as versões 2010 e 2012 (atual), apenas correções no 
%% texto e algumas definições para impressão frente e númeração de páginas
%% para trabalhos com mais de 100 páginas.
%% Dito isso, NÃO GARANTO que a classe esteja totalmente de acordo com
%% qualquer versão do MDT.
%%
%% Opções para a classe do documento:
%%  - projtg: Projeto de TG
%%  - tg: Trabalho de Graduação
%%  - espec: Monografia de Especialização
%%  - tese: Tese de Doutorado
%%  - diss: Dissertação de Mestrado
%%
%% Este documento e um exemplo de Projeto de TG, para outras funcionalidades
%% olhar no mdtufsm.cls :)
%%
%% É necessário seguir esta sequência para compilar o documento:
%%	- compilar LaTeX
%%	- compilar BibTeX
%%	- compilar LaTeX
%%	- compilar LaTeX
%%
%% O warning "Underfull \vbox (badness 10000) detected at line XX" pode ser
%% ignorado sem problemas.
%%=============================================================================

\documentclass[projtg]{mdtufsm}

\usepackage[T1]{fontenc}
\usepackage[utf8]{inputenc}
% para funcionar corretamente o tamanho das fontes da capa
\usepackage{fix-cm}
% pacote para usar fonte Adobe Times e cores			
\usepackage{times, color, xcolor}
% pacote para importar figuras
\usepackage{graphicx}
% pacotes matemáticos
\usepackage{amsmath, latexsym, amssymb}
\usepackage{tabu}
\usepackage[inner=30mm, outer=20mm, top=30mm, bottom=20mm]{geometry} 
% só é necessário para o lorem ipsum
\usepackage{lipsum}
% hidelinks disponível no pacote hyperref a partir da versão 2011-02-05  6.82a
\usepackage[%hidelinks%, 
            bookmarksopen=true,linktoc=none,colorlinks=true,
            linkcolor=black,citecolor=black,filecolor=magenta,urlcolor=blue,
            pdftitle={Título do Trabalho},
            pdfauthor={Autor},
            pdfsubject={Assunto (TG, Projeto, etc.)},
            pdfkeywords={keyword1, keyword2}
            ]{hyperref}


%%=============================================================================
%% Trampa para corrigir o bug do hyperref que redefine o caption das figuras e das
%% tabelas, n�o colocando o nome ``Figura'' antes do n�mero do mesmo na lista
%%=============================================================================

\makeatletter

\long\def\@caption#1[#2]#3{%
  \expandafter\ifx\csname if@capstart\expandafter\endcsname
                  \csname iftrue\endcsname
    \global\let\@currentHref\hc@currentHref
  \else
    \hyper@makecurrent{\@captype}%
  \fi
  \@ifundefined{NR@gettitle}{%
    \def\@currentlabelname{#2}%
  }{%
    \NR@gettitle{#2}%
  }%
  \par\addcontentsline{\csname ext@#1\endcsname}{#1}{%
    \protect\numberline{\csname fnum@#1\endcsname ~-- }{\ignorespaces #2}%
  }%
  \begingroup
    \@parboxrestore
    \if@minipage
      \@setminipage
    \fi
    \normalsize
    \expandafter\ifx\csname if@capstart\expandafter\endcsname
                    \csname iftrue\endcsname
      \global\@capstartfalse
      \@makecaption{\csname fnum@#1\endcsname}{\ignorespaces#3}%
    \else
      \@makecaption{\csname fnum@#1\endcsname}{%
        \ignorespaces
        \ifHy@nesting
          \expandafter\hyper@@anchor\expandafter{\@currentHref}{#3}%
        \else
          \Hy@raisedlink{%
            \expandafter\hyper@@anchor\expandafter{%
              \@currentHref
            }{\relax}%
          }%
          #3%
        \fi
      }%
    \fi
    \par
  \endgroup
}

\makeatother

\title{Sistema de análise de dados do Programa Nacional de Formação Continuada de Conselheiros Municipais de Educação - Pró-Conselho/UFSM}

\author{Machado Lunardi}{Gabriel}
\course{Sistemas de Informação}
\altcourse{curso de Sistemas de Informação}
\institute{Centro de Tecnologia}
\degree{Bacharel em Sistemas de Informação}
\trabalhoNumero{}
\advisor[Prof.ª]{Dr.ª}{Winck}{Ana T.}

\date{15}{Agosto}{2014}

% Exemplos de keywords
\keyword{Classificação Multirrótulo} 
\keyword{Aprendizado de Máquina}
\keyword{Mineração de Dados}

\begin{document}
\maketitle
\setlength{\baselineskip}{1.5\baselineskip}

\chapter{Introdução}

O Programa do Pró-Conselho é proposto pelo Ministério da Educação (MEC), consiste em um Curso de Extensão a Distância de Formação Continuada de Conselheiros Municipais de Educação. O curso configura-se como iniciativa da Secretaria de Educação Básica (SEB) que visa fortalecer os Sistemas de Ensino e as instâncias políticas e sociais tal como é o Conselho Municipal de Educação (CME). Tem carga horária de 160h, ofertado via internet, em ambiente virtual de aprendizagem (plataforma Moodle) ministrado pela Universidade Federal de Santa Maria em parceria com a Coordenação do Programa Nacional de Capacitação de Conselheiros Municipais de Educação – Pró-Conselho, SEB/MEC. Dentre os conteúdos discutidos e estudados estão: Educação e Tecnologia, Concepção, Estrutura e Funcionamento do CME, Conselho Municipal e as Políticas Públicas. O público alvo é formado por Conselheiros Municipais de Educação e técnicos das secretarias de educação dos municípios onde ainda não existam Conselhos Municipais de Educação. 

 O Pró-Conselho/UFSM é responsável pela oferta do curso no estado do Rio Grande do Sul - RS, lotado no Centro de Educação da UFSM, estando em sua 2ª edição. A equipe é composta por: coordenação geral; coordenação adjunta local responsável por contatar cursistas (alunos) pessoalmente ou por telefone; coordenação adjunta pedagógica para resolver questões de aprendizado; um professor supervisor com o intuito de orientar e guiar a tutoria; tutores que agem diretamente com os cursistas; apoio de Informática encarregado da gestão tecnológica; apoio administrativo incumbido das tratativas administrativas. Vale ressaltar que o projeto está diretamente ligado aos grupos de pesquisa dessa área, pois serve como objeto de estudo. 

Nessa perspectiva, como uma das atividades do curso, a equipe gestora formulou um questionário composto por perguntas abertas e fechadas buscando delinear o perfil e a realidade dos CMEs do RS. Tal questionário foi utilizado como requisito inicial no desenvolvimento de um banco de dados o qual receberá as respostas através de um sistema {\it web} a ser desenvolvido para coleta, visualização e posterior análise dos dados. É importante destacar que existe um sistema, desde 2003, para o mesmo fim, de âmbito nacional, chamado Sistema de Informações dos Conselhos Municipais de Educação (SICME). No entanto, o sistema encontra-se em manutenção segundo consta no portal eletrônico do MEC.

Dentre as tecnologias para análise de dados, destaca-se a mineração de dados. A mineração de dados {\it Data Mining} é uma etapa do processo de KDD {\it Knowledge Discovery in Databases} - Descoberta de Conhecimento em Bases de Dados buscando por padrões em grandes bases de dados. Para \cite{Thuraisingham99} mineração de dados é o processo de proposição de várias consultas, extração de informações úteis, padrões e tendências a partir de grande quantidade de dados armazenada em banco de dados. 

Tendo em vista o conhecimento sobre a mineração de dados e desenvolvimento de sistemas, objetiva-se construir um sistema {\it web} capaz de prover a coleta, gerenciamento, visualização e a análise os dados buscando delinear o perfil e a realidade dos CMEs no RS.

A escolha do contexto do trabalho justfica-se através dos fatores: inatividade do SICME e suas incongruências devido a sua grande abrangência; necessidade de um recurso computacional, por parte da equipe, para coletar e analisar os dados de forma automatizada e a oportunidade de consolidar conhecimentos em mineração de dados em ambientes reais.

Como consequência deste trabalho pretende-se contribuir com uma metodologia de análise e visualização de dados automatizada para o Pró-Conselho/UFSM e, também, como uma sugestão de iniciativa, por outros estados, à descentralização do SICME. Assim, busca-se contribuir para o fortalecimento dos CME no RS e na sociedade com leis mais bem definidas a partir das informações descobertas. 

\chapter{Objetivos}
\section{Objetivo Geral}

O objetivo geral deste trabalho é desenvolver um sistema para coleta, gerenciamento e posterior visualização e análise de dados utilizando técnicas relacionadas à mineração de dados, sobre os dados obtidos das respostas dos cursistas do Pró-Conselho/UFSM. 
	
\section{Objetivos Específicos}


\chapter{Justificativa}


\chapter{Revisão de Literatura}


\chapter{Metodologia}

    
\chapter{Cronograma}
\begin{tabu} to 0.9\linewidth{|X[6]|X|X|X|X|X|}
	\hline
	& AGO & SET & OUT & NOV & DEZ \\
	\hline
	Definição do projeto & X & & & & \\
	\hline
	Revisão de Literatura & X & X & & & \\
	\hline
	Apresentação do andamento do TG & & & X & & \\
	\hline
	Elaboração da parte escrita & & X & & & \\
	\hline
	Reuniões de acompanhamento & X & X & X & X & X \\
	\hline
	Apresentação do TG para a banca & & & & & X \\
	\hline
\end{tabu}
 
\setlength{\baselineskip}{\baselineskip}

% Referências
\bibliographystyle{abnt}
\bibliography{referencias}

\end{document}
